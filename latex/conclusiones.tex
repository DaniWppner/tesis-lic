En este trabajo realizamos una apreciación de las funcionalidades de Manticore, al mismo tiempo que un desglose de su arquitectura interna y su API programable.
Dicho análisis fue relevante a la hora de poder idear una estrategia para utilizar Manticore en la construcción de EPAs, y en la indagación de los tiempos de ejecución de la herramienta.

Desarrollamos un algoritmo para la construcción de EPAs en el caso de tener acceso a las precondiciones de los métodos pero no al invariante, orientado al uso de ejecución simbólica y en particular, Manticore.
Vimos las limitaciones de este nuevo algoritmo frente a cierto tipo de contratos, y pudimos corroborar tanto la facilidad con la que se lo puede hacer fallar como corroborar su funcionamiento en benchmarks de smart contracts utilizados anteriormente.

Pudimos contrastar la performance en tiempo del algoritmo novedoso contra el algoritmo de construcción de EPAs clásico para ejemplos simples, y obtuvimos que el algoritmo novedoso presentaba grandes mejoras sobre el clásico al implementar ambos con Manticore, aunque el tiempo total de ejecución de este algoritmo seguía sin resultar satisfactorio.

Al buscar los principales culpables de la alta cantidad de tiempo empleado por el prototipo implementado, en busca de optimizaciones, identificamos que la etapa responsable de la mayoría del tiempo de ejecución	era la de ejecución simbólica de métodos de los contratos.
Esta etapa asimismo resultaba la más difícil de analizar (y por ende de optimizar), debido a cómo se encontraba programada.
Por otro lado, realizamos un análisis evaluando si la mayoría del tiempo era consumido por Manticore mismo o por los programas de SMT solving, y obtuvimos resultados que sugieren que este fuera consumido por los procesos de Manticore.

Consideramos que áun existen algunas líneas que se pueden explorar en términos de optimización en tiempo de la herramienta implementada.
Por un lado, podría indagarse más profundo en el comportamiento de Manticore, intentando observar cuáles son las consultas que realiza a los programas de SMT solving.
Por otro lado, explorar el impacto que tiene la explosión de estados producida por la presencia de múltiples \texttt{accounts} durante los análisis en el tiempo de ejecución de Manticore puede proveer un mayor entendimiento, además de ser un punto de partida para explorar técnicas que supriman esta misma explosión.

Creemos que una línea de investigación futura es el uso de los callbacks de la API de manticore para realizar un análisis en el que se avanza de a un único \textcolor{cyan}{\texttt{state}} por vez, más similar a las estrategias clásicas de construcción de EPAs.
Por otro lado, es posible que la capacidad de emular blockchains enteras de Manticore le permita encontrar propiedades en las abstracciones que le resulten más difíciles a las otras herramientas.
Creemos que analizar esta hipótesis podría ser valioso para evaluar el valor de la estrategia presentada en este trabajo.

\section{Trabajo relacionado}
Desde hace un tiempo se emplea abstracción por predicados, similares a las de este trabajo, para realizar verificación de programas \cite{predicate-abstraction-for-verification}.
Cuando el objetivo es verificación, el nivel de abstracción, el tamaño del modelo resultante y la facilidad de vincular las abstracciones generadas con el programa original varían.
Abstraer al llamado de función permite generar abstracciones mucho mas amenas para la validación, aunque agrupar tanto el comportamiento las hace menos útiles para la verificación.

La generación de EPAs es una técnica similar a la generación de typestates \cite{high-level-protocols-for-low-level-software} de un programa.
Estas abstracciones pensadas para la verificación son menos permisivas que las EPAs, y sólo permiten trazas válidas en el artefacto original.
Otra técnica similar es la inferencia de interfaces \cite{permissive-interfaces}.
Existen técnicas para inferir modelos del comportamiento a partir de trazas de ejecución, infiriendo especificaciones para sucesivamente generar nuevos casos de test \cite{inference-by-traces}.

En el contexto de contratos inteligentes, existen múltiples herramientas ya mencionadas dedicadas a encontrar vulnerabilidades de seguridad \cite{oyente} \cite{echidna} \cite{manticore} \cite{pakala} \cite{teether} o verificación de propiedades \cite{solidity-to-CPN} \cite{formal-analysis-based-on-CPN} \cite{verisol}.
Otros acercamientos \cite{VeriSolid} construyen contratos inteligentes correctos a partir de especificaciones.
Con respecto a herramientas orientadas a la validación, existen visualizadores \cite{surya} \cite{slither} de grafos de control de flujo y herencia de contratos en Solidity, pero que no visualizan funcionalidad o comportamiento como lo hacen las abstracciones de este trabajo.

