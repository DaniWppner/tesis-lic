La mayor evidente diferencia entre el algoritmo presentado en la seccion \ref{sec:algoritmo-aternativo} y el algoritmo clásico es que la manera en la que explora las transiciones de la EPA lo obliga a considerar solo ciertos caminos (a alto nivel) de transacciones para generar transiciones en la EPA, en lugar de transiciones entre estados completamente abstractos.
Consideremos el siguiente ejemplo:




%A pesar de que Manticore permita el uso de valores simbólicos para el parámetro \texttt{(msg.sender)} las limitaciones de la herramienta exigen que se defina cada account individualmente para que sea considerada en el estado de la blockchain.
Las limitaciones de la herramienta exigen que el parámetro \texttt{(msg.sender)}, a pesar de que pueda ser simbólico, se corresponda con una \textit{account} definida explícitamente por el usuario.
Por esto en la simulación contamos con un número fijo de \textit{accounts}, asignando variables simbólicas únicamente la información asociada a ellas (Dirección, Balance, etc).

Al evaluar el funcionamiento del prototipo buscamos responder las siguientes preguntas:
\begin{enumerate}
    \item ¿Son correctas las EPAs que genera el prototipo?
    \item ¿Cuál es su performance en contratos inteligentes reales?
\end{enumerate}
Para responder estas preguntas, pusimos el prototipo a prueba contra algunos contratos provenientes de Microsoft Azure Blockchain Workbench \cite{azure-benchmark}.
Este conjunto de contratos había sido utilizado anteriormente por Godoy et al. \cite{predicate-abstraction-for-smart-contract-validation}, por lo que convenientemente ya contamos con las EPAs correspondientes.
Ejecutamos el prototipo cinco veces, obteniendo el promedio de su tiempo de ejecución sobre los contratos seleccionados y luego corroboramos que la EPA generada fuera isomórfica a la obtenida en los estudios anteriores.
%Además, pusimos a prueba la cantidad de accounts necesarias para el funcionamiento del prototipo.
Los resultados intermedios indicaron que para los contratos propuestos era suficiente realizar la simulación con dos \textit{accounts}, por lo que utilizamos esa cantidad para los experimentos.
Los resultados de esta experimentación se ven resumidos en la tabla \ref{tab:resultados}.
El prototipo generó EPAs correctas en todos los casos.
Sin embargo, el tiempo de cálculo es de entre media y casi tres horas, considerando incluso que los ejemplos utilizados son relativmente pequeños.
Algunos resultados intermedios indican que este tiempo es consumido principalmente por Manticore para la generación de \textit{path conditions}.
Debido a la rigurosidad con la que emula Manticore el comportamiento de la blockchain, la herramienta demora incluso para la ejecución simbólica de transiciones sencillas.

\vspace{-2.2em}

\begin{table}
    %\begin{adjustwidth}{-0.5in}{-0.5in}% adjust the L and R margins by 1 inch
    \caption{Resumen de la experimentación. \textbf{LOC} es cantidad de lineas de código, \textbf{Tiempo de ejecución} es el promedio del tiempo de ejecución medido en minutos, $\boldsymbol{\sigma}$ es el desvío estandard medido en segundos y \textbf{¿Es correcto?} indica si la EPA generada es isomorfa con la provista anteriormente.}\label{tab:resultados}
    \begin{tabular}{|l|l|l|l|l|}
        \hline
        Contrato                  & LOC & Tiempo de ejecución (min) & $\sigma$ (s) & ¿Es correcto? \\
        \hline
        DefectiveComponentCounter & 33  & $29$                      & $7$          & Sí            \\
        SimpleMarketplace         & 66  & $186$                     & $800$        & Sí            \\
        BasicProvenance           & 48  & $40$                      & $6$          & Sí            \\
        RoomThermostat            & 48  & $138$                     & $600$        & Sí            \\
        \hline
    \end{tabular}
    %\end{adjustwidth}
\end{table}
