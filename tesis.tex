%!LW recipe=pdflatex -> bibtex -> pdflatex * 2

\documentclass[11pt,a4paper,twoside]{tesis}
% SI NO PENSAS IMPRIMIRLO EN FORMATO LIBRO PODES USAR
%\documentclass[11pt,a4paper]{tesis}

\usepackage{graphicx}
\usepackage[utf8]{inputenc}
\usepackage[spanish]{babel}
\usepackage[left=3cm,right=3cm,bottom=3.5cm,top=3.5cm]{geometry}

\usepackage{url}

\usepackage{amsmath}
\usepackage{amsthm}
\usepackage{amssymb}
\usepackage{dsfont}
\usepackage{bold-extra}
\usepackage{biblatex}
\usepackage{csquotes}
\usepackage{algpseudocode}
\usepackage{algorithm}
\usepackage{caption}
\usepackage{subcaption}
\usepackage{svg}
\newcommand{\pluseq}{\mathrel{+}=}
\usepackage{chngpage}
\usepackage[skins]{tcolorbox}
\newcommand\newsubcap[1]{\phantomcaption%
       \caption*{\figurename~\thefigure(\thesubfigure): #1}}
\floatname{algorithm}{Algoritmo}
\usepackage{xcolor}

%THEOREM STYLES
\theoremstyle{definition}
\newtheorem{definition}{Definition}[section]

\addbibresource{bibliografia.bib}
\begin{document}

%%%% CARATULA

\def\autor{Daniel Wappner}
\def\tituloTesis{Construcción de Enabledness Preserving Abstractions para smart contracts mediante ejecución simbólica}
\def\runtitulo{Construcción de Enabledness Preserving Abstractions para Smart Contracts mediante ejecución simbólica}
\def\runtitle{Construcción de Enabledness Preserving Abstractions para smart contracts mediante ejecución simbólica}
\def\codirector{Diego David Garbervetsky}
\def\director{Javier Godoy}
\def\lugar{Buenos Aires, 2024}
\input{caratula}

%%%% ABSTRACTS, AGRADECIMIENTOS Y DEDICATORIA
\frontmatter
\pagestyle{empty}
%\begin{center}
%\large \bf \runtitulo
%\end{center}
%\vspace{1cm}
\chapter*{\runtitulo}

\noindent Los smart contracts son programas inmutables que se despliegan en una blockchain.
Dado que a menudo manejan activos de alto valor real, su verificación y validación antes de desplegarlos es de gran importancia.
Por esta razón, es una práctica común contratar empresas de seguridad especializadas para auditar el código de los smart contracts.
Sin embargo, se han explotado numerosas vulnerabilidades en los últimos años provocando pérdidas a miles de personas.

Las \textit{Enabledness Preserving Abstractions} (EPAs), son máquinas de estado finitas que abstraen el comportamiento de artefactos de código, basándose en predicados sobre la habilitación de los métodos disponibles.
En general, han resultado útiles como herramienta para la validación de código tanto contra especificaciones formales como contra modelos informales o ``mentales'' del comportamiento esperado.

Presentamos un protitpo que genera EPAs de contratos inteligentes a partir de código fuente, haciendo uso y extensión de una herramienta open source de ejecución simbólica dinámica: ``Manticore''.
Discutimos las optimizaciones implementadas y comparamos el prototipo desarrollado con otras estrategias alternativas.

\bigskip

\noindent\textbf{Palabras claves:}  Contratos inteligentes, Ejecución simbólica, Construcción de abstracciones, Validación, Modelado, Solidity, Análisis estático.

%\cleardoublepage
%%\begin{center}
%\large \bf \runtitle
%\end{center}
%\vspace{1cm}
\chapter*{\runtitle}

\noindent In English

\bigskip

\noindent\textbf{Keywords:} (no less than 5). % OPCIONAL: comentar si no se quiere

%\cleardoublepage
%\chapter*{Agradecimientos}

Antes que nada, quería agradecer a los jurados de esta tesis: Sebastián Uchitel y Ariel Waisbein por tomarse el tiempo de leer y evaluar este trabajo.

Quería agradecer al LaFHIS en su conjunto, no solo por abrirme las puertas para realizar este trabajo y otras oportunidades que me ofrecieron, sino también por brindar un ambiente tan lindo para realizar este trabajo.  

Me gustaría agradecer también a todo el apoyo que recibí para poder llegar hasta acá en la carrera.
A les docentes que año tras año no sólo dictan las clases cada vez más masivas sino que además preparan material, exámenes y las materias; y administratives que que resuleven tantos problemas en horas extra para que la gente pueda estudiar en la universidad públca.
A todas las personas involucradas en mi camino personal que recorrí por la facultad, y a mis amigos y compañeros de cursada en particular: Erik, Ale, Abrol, Laureano, Charles, Dani, Blas, Ivo, Franco, Luciana, Sujo, Luciano, Juli, Fran, Yuri y Javi.

A mis amigues de cuando no estoy en la facultad, gracias por estar dispuestos a hacer tantas cosas distintas juntos, y en esta vuelta a bancarme con que no pueda ir a ningún lugar nunca. Los aprecio muchísimo Lucio, Eliseo, Iván, Nico, Lucas, Feli, Marto, Aylen, Michu, Bruno.

Javi y Diego, muchísimas gracias.
Me ayudaron muchísimo al dirigirme la tesis, fueron siempre personas increíbles, y realmente me hicieron sentir incluido un montón.
Realmente siempre están disponibles a cualquier cosa y cada duda, problema o detalle que puedas tener.

Quiero además agradecer a mi familia por acompañarme hasta acá.
A quienes me acompañaron desde chiquito, abuelos Sasha y Angel; abuelas Didi y María; y Marta.
A todos mis primos y tíos que me encanta ver.
A los más cercanos, gracias Papá, Mamá, Marcos, Meli y Cami. 
 % OPCIONAL: comentar si no se quiere

%\cleardoublepage
%\input{dedicatoria.tex}  % OPCIONAL: comentar si no se quiere

\cleardoublepage
\tableofcontents

\mainmatter
\pagestyle{headings}

%%%% ACA VA EL CONTENIDO DE LA TESIS

\chapter{Introduccion}
Las redes de blockchain siguen un protocolo para manejar un registro distribuido ``confiable'' de los hechos.
La participación en estas redes es por diseño anónima y descentralizada, y hace gran hincapié en la resistencia del sistema a ataques criptográficos.
Una vez registrada, resulta casi imposible  alterar información introducida en una blockchain.
A fines prácticos, una transacción en una de estas bases de datos suele ser considerada eterna e inmutable.
Adicionalmente, la cadena de bloques propiamente dicha, es decir, el historial de la información registrada, es transparente y pública \cite{bitcoin-overview} \cite{survey-of-blockchain-security}.

Desde el origen de Bitcoin en 2009, estas tecnologías se utilizaron para construir un ``libro contable'' distribuido  de transacciones financieras \cite{survey-of-blockchain-security}.
Más adelante, Ethereum es pionera en 2014 \cite{ethereum-white-paper} y se torna desde entonces la más popular de las redes que utilizan su blockchain para soportar de manera distribuida la ejecución de software.
A los programas que se ejecutan en las blockchain se los conoce como contratos inteligentes o \textit{smart contracts} y la capacidad de computarlos se integra íntimiamente al diseño de las mismas.
De hecho, los protocolos de consenso de las blockchains que soportan smart contracts quedan firmemente atados al modelo de cómputo elegido.
Como ejemplo, la red Ethereum provee la \emph{Ethereum Virtual Machine}\cite{ethereum-yellow-paper}, una máquina de pila de profundidad finita, y para su programación desarrolló Solidity\cite{solidity}, un lenguaje de programación compilado curly-brace.
Otra blockchain popular que soporta smart contracts es Algorand, mediante la \textit{Algorand Virtual Machine} \cite{algorand-avm}, otra máquina de pila junto a TEAL, el lenguaje de programación tipo assembly correspondiente.

%El código y estado de los programas es parte de la información guardada en la blockchain.
Una propiedad de las blockchains como sistema de cómputo distribuido es que el modo en el que los contratos inteligentes son incluidos en la blockchain garantiza la seguridad de su ejecución, en el sentido de que puede asegurarse que el resultado obtenido será el mismo que el generado por un código fuente que los usuarios pueden conocer \cite{survey-of-blockchain-security}.
La inmutabilidad de las transacciones registradas asegura que los contratos no pueden ser modificados, lo que además garantiza a los usuarios que el comportamiento de los contratos se mantendrá siempre estable.
Por otro lado, dado que los cambios en el estado del programa son registrados en la blockchain, estos también se consideran irreversibles.
Esto, a pesar de ciertas garantías de estabilidad que les otorga a los usuarios, significa que los defectos en la implementación de los contratos inteligentes no pueden ser reparados, y las transacciones no deseadas originadas de estos defectos no pueden ser revertidas.
Si se quieren evitar defectos, la verificación y validación de los contratos antes de publicarlos en la red es de suma importancia.
De hecho, los ataques a contratos inteligentes para abusar \textit{bugs} han causado históricamente pérdidas materiales a miles de personas \cite{DAO}.

Por esto, es común en la industria contratar a auditores independientes para la validación de contratos inteligentes,
quienes a menudo usan herramientas para facilitar su trabajo y a menudo cuentan sólo con especificaciones informales del comportamiento deseado.
En estas situaciones, una de las tantas técnicas útiles para la validación es  la construcción de una máquina de estados finita que abstraiga el comportamiento del contrato, destacando sólo ciertas propiedades interesantes \cite{predicate-abstraction-for-smart-contract-validation}.
Estas máquinas abstraen al nivel de ``llamado a función'', y están fuertemente basadas en las \textit{EPAs} (Enabledness Preserving Abstraction) \cite{de-caso-epa} que son máquinas de estado que describen las posibles secuencias de llamados a funciones del contrato.
En el trabajo presentado por Godoy et al. en 2022 \cite{predicate-abstraction-for-smart-contract-validation} utilizan un enfoque donde construyen EPAs a partir de código fuente traduciendo manualmente las pre y post condiciones de los contratos inteligentes a un lenguaje intermedio, Alloy \cite{alloy}.
Más tarde en 2023 Torres at al. \cite{torres} presentó un trabajo en el que desarrollan un prototipo que genera las abstracciones a partir de contratos inteligentes programados en Solidity \footnote{El prototipo funciona para versiones de Solidity anteriores a ¿?} de manera automática mediante model checking.

\section{Objetivos}

Nos interesa explorar otras alternativas para la construcción de EPAs para contratos inteligentes implementados en Solidity de manera automática.
Buscamos implementar un prototipo que se base en ejecución simbólica utilizando Manticore, una herramienta open source de ejecución simbólica dinámica con soporte tanto para la EVM como código nativo desarrollada por trailofbits \cite{manticore}.
Esperamos generar abstracciones correctas y de manera eficiente.
Además, Manticore respalda la simulación de una blockchain completa, manteniendo el estado de múltiples contratos y usuarios dentro de la red.
Buscamos aprovechar estas cualidades para detectar comportamientos más complejos en las abstracciones generadas.

\section{Trabajo relacionado}


\section{Ejemplo motivador}

Supongamos que un

\chapter{Motivación}

\chapter{Conceptos Preliminares}
\section{Ejecución simbólica dinámica}

\section{Contratos Inteligentes y Solidity}
Una blockchain es una base de datos pública, distribuida e inmutable.
En Ethereum cada transacción registrada se refiere a propiedades sobre \textit{direcciones} en la base de datos como el balance de criptomonedas o el estado de un contrato inteligente.
Las transacciones se agrupan en conjuntos de \textit{bloques}, que son la unidad mínima con la que se actualiza el estado de la blockchain.
El historial de bloques agregados a la blockchain (y por ende, las transacciones ejecutadas) es público, y el estado actual de la red puede calcularse a partir de este historial.
Además, los protocolos de consenso utilizados por la red le otorgan garantías de seguridad de que ningún actor podrá forzar el ingreso de información falsificada a ella \cite{protocolos-consenso}.

Como ya dijimos, Ethereum y otras blockchain mantienen un estado de cómputo distribuido, y permiten modificar este estado mediante aplicaciones programadas llamadas contratos inteligentes.
Los contratos inteligentes se corresponden con direcciones dentro de la base de datos que es la blockchain, de la misma forma que se corresponden con direcciones los usuarios tradicionales.
De hecho, en Ethereum los contratos inteligentes y los usuarios tradicionales cuentan con dos tipos de \textit{accounts} distintas, que pueden pertenecer a usuarios humanos por un lado o a contratos inteligentes por el otro .
En el caso de los usuarios humanos el almacenamiento que le corresponde a cada dirección guarda sólo información sobre el balance de criptomonedas, mientras que un contrato inteligente utiliza dos campos adicionales: el \textit{codeHash}, que es utilizado para obener el código ejectuable del contrato y el \textit{storageRoot}, que es necesario para conocer una especie de memoria persistente exclusiva del contrato.
Ambos tipos pueden recibir o enviar mensajes y Ether\footnote{Ether es la criptomoneda utilizada por Ethereum}.
La diferencia en los dos tipos de accounts radica en qué ocurre cuando la account es seleccionada como destinataria de una transferencia de fondos o mensaje.
Mientras que para un usuario humano simplemente se registra el suceso en la blockchain, en el caso de un contrato inteligente, este es el momento en el que el programa se ejecuta, agregando el mensaje que disparó la ejecución al contexto de la misma.
Para poder incluir la transacción en el próximo estado de la blockchain, el minero\footnote{Los ``mineros'' son quienes participan en el protocolo de la blockchain y registran las transacciones publicadas por usuarios en bloques} que registre la transacción debe levantar la EVM y simular la ejecución del contrato para conocer su  resultado y estado final.

La EVM es una máquina de pila casi turing completa.
El lenguaje assembly asociado cuenta con instrucciones para números enteros, instrucciones para control de flujo y otras instrucciones referentes a particularidades de la ejecución en blockchain, como el contenido de las transacciones de la misma \cite{evm-opcodes}.
A pesar de que el conjunto de programas expresablas en el lenguaje de la EVM es turing completo, la cantidad de operaciones que se pueden ejecutar dentro de una misma transacción están limitadas por la cantidad de \textit{gas} que esté dispuesto pagar el remitente de la transacción al minero que la publique.
De hecho, existe una cota superior a la cantidad de gas que tiene permitido consumir un contrato, lo que significa que una ejecución dada de la EVM siempre termina en un número finito de instrucciones.

En la actualidad, existen lenguajes utilizados para programar contratos en la red Ethereum.
Sin embargo que más adopción que tiene actualmente para la red Ethereum es Solidity.
Es un lenguaje imperativo curly-brace que provee una sintaxis similar a la de las clases en lenguajes orientados a objetos.
La sintaxis empleada para definir los contratos incluye variables de estado, un método constructor, métodos internos (ejecutables sólo por el mismo contrato), y  un conjunto de métodos externos que representan la interfaz con otros contratos y el mundo con la que los usuarios pueden interactuar.
La ejecución de los métodos externos es disparada cuando los contratos son receptores de un mensaje en la blockchain, eligiendo cuál método ejecutar en base al contenido del mensaje \cite{??}. %Algo de documentación de Solidity

\subsection{Ejemplo}
En la figura \ref{fig:solidity-example} presentamos un ejemplo de un programa escrito en Solidity, \texttt{SimpleMarketplace}\footnote{Este contrato es extraido de los contratos de ejemplo ``Mircrosoft Azure Blockchain Workbench'' \cite{azure-benchmark}}, el cual implementa un mecanismo simple para vender un bien.
Analizando línea a línea del contrato, se define en la línea 2 un tipo que tiene tres valores posibles (``\texttt{StateType}'') que representa estados de completitud de la venta.
Las líneas 3 a 8 definen las variables de estado del contrato, entre las que tenemos variables de tipo \texttt{int}, \texttt{StateType} y \texttt{address}, indicando además mediante la palabra clave \textcolor{blue}{\texttt{public}} que resultan accesibles por contratos externos.
El tipo \texttt{address} utilizado en \texttt{InstanceOwner} y \texttt{InstanceBuyer} consiste en valores enteros de 20 bytes de tamaño, y se utiliza para representar direcciones en la blockchain.

\begin{lstlisting}[language=Solidity, label={fig:solidity-example}, caption={Contrato Inteligente \texttt{SimpleMarketplace} en Solidity},captionpos=b]
pragma solidity >=0.4.25 <0.9.0;
pragma experimental ABIEncoderV2;

contract SimpleMarketplace {

    enum StateType {ItemAvailable, OfferPlaced, Accepted}
    address public InstanceOwner;
    string public Description;
    int public AskingPrice;
    StateType public StateEnum;  
    address public InstanceBuyer;
    int public OfferPrice;

    constructor(string memory description, int price, address sender) public
    {
        InstanceOwner = sender;
        AskingPrice = price;
        Description = description;
        StateEnum = StateType.ItemAvailable;
    }

    function MakeOffer(int offerPrice) public
    {
        require (offerPrice != 0 && StateEnum == StateType.ItemAvailable && InstanceOwner != msg.sender);
        InstanceBuyer = msg.sender;
        OfferPrice = offerPrice;
        StateEnum = StateType.OfferPlaced;
    }

    function Reject() public
    {
        require (StateEnum == StateType.OfferPlaced && InstanceOwner == msg.sender);
        StateEnum = StateType.ItemAvailable;
    }

    function AcceptOffer() public
    {
        require (StateEnum == StateType.OfferPlaced && msg.sender == InstanceOwner);
        StateEnum = StateType.Accepted;
    }
}
\end{lstlisting}

Observando el método \textcolor{blue}{\texttt{constructor}} de \texttt{SimpleMarketplace}, que es el método que siempre se llama al desplegar una instancia del contrato, vemos que define el objeto que se planea vender, indicando la descripción, el precio del producto y la dirección de su dueño actual \footnote{En la definición de \textcolor{blue}{\texttt{constructor}}, la palabra clave \textcolor{blue}{\texttt{memory}} indica qué estrategia debe usarse para sostener en memoria el string que entra por parámetro. Otras opciones, como \textcolor{blue}{\texttt{calldata}}, impactan en el costo de gas de la función.}.
Luego, las líneas 20, 28 y 34 indican precondiciones de los métodos \textcolor{orange}{\texttt{MakeOffer}}, \textcolor{orange}{\texttt{Reject}} y \textcolor{orange}{\texttt{AcceptOffer}}.
En ellas, la expresión \textcolor{blue}{\texttt{msg.sender}} se refiere a la dirección desde la que se envió el mensaje que disparó la ejecución actual del contrato.
Si prestamos atención y debido a que el valor inicial de \texttt{SateEnum} es \texttt{ItemAvailable}, el único método que está permitido llamar inmediatamente después del constructor es \textcolor{orange}{\texttt{MakeOffer}}.
Mediante este, un potencial comprador puede realizar ofertas sobre el producto indicando un precio, que luego el dueño puede aceptar o rechazar.
Si el dueño original no está satisfecho con la oferta puede rechazarla llamando a \textcolor{orange}{\texttt{RejectOffer}}, regresando al estado incial en el que se aceptan nuevas ofertas.
Si eventualmente al dueño le interesa la última oferta realizada y la acepta, llama a \textcolor{orange}{\texttt{AcceptOffer}} y termina satisfactoriamente la ejecución del contrato, ya que
a partir de ese estado ningún otro método se encuentra habilitado. Los detalles de la venta realizada permanecen expuestos en las variables públicas del contrato, que a pesar de quedarse bloqueado a nuevos llamados a métodos permanece visible en la blockchain.


\section{Enabledness-Preserving Abstractions}


Una EPA es un Labeled Transition System (LTS) finito que busca abstraer el comportamiento de un contrato inteligente agrupando los estados del contrato en base a cuáles de sus metodos están habilitados \cite{de-caso-epa}.
Las transiciones en estos LTS representan el llamado a una función del contrato, indicando cómo un llamado a una función puede transformar el contrato de un estado abstracto a otro.

Comenzando por un ejemplo, en la figura \ref{fig:epa-example} presentamos la EPA del contrato \texttt{Simple\-Marketplace}.
El estado inicial, denominado \textbf{A}, está etiquetado \texttt{init} e indica el estado ``vacío'' previo al llamado al constructor del contrato.
Lo que podemos ver es que luego de ejecutar el método constructor se transiciona en la EPA a un único otro estado, al que llamamos \textbf{B}.
Allí, la etiqueta \texttt{\_MakeOffer} indica que \textcolor{orange}{\texttt{MakeOffer}} es el único método que se encuentra habilitado en \textbf{B}.
La única transición desde \textbf{B}, etiquetada \textcolor{orange}{\texttt{MakeOffer}}, indica lo que ocurre cuando se ejecuta ese método desde ese estado. Como vemos, seguir esa transición nos traslada  al estado \textbf{C}, que es un estado del contrato en el que solamente \textcolor{orange}{\texttt{AcceptOffer}} y \textcolor{orange}{\texttt{Reject}} se encuentran habilitados.
Luego, desde el estado \textbf{C} podemos ejecutar cualquiera de los dos métodos; la transición por \textcolor{orange}{\texttt{Reject}} nos llevará de vuelta al estado \textbf{B} y la transición por \textcolor{orange}{\texttt{AcceptOffer}} nos llevará al estado final \textbf{D}.
Este último, etiquetado ``\texttt{vacio}'' indica que ningún método se encuentra habilitado, por lo que representa el fin forzoso de la ejecución.



\begin{figure}
    \includegraphics[width=0.3\textwidth]{figs/simple-merketplace-epa.png}
    \caption{EPA de \texttt{SimpleMarketplace}. Las etiquetas en los estados indican los métodos que se encuentran habilitados. Las etiquetas en las transiciones indican el método por el que ocurre la transición.}
    \label{fig:epa-example}
\end{figure}

\section{Modelo Formal}
Dado que trabajaremos sobre el código fuente de un contrato escrito en Solidity, nos interesa formalizar qué aspectos del contrato consideraremos relevantes.
Para obtener una discusión más detallada de estas formalizaciones de los artefactos de código, referirse a De Caso et. al.  \cite{de-caso-epa}.
Asimismo, la abstracción presentada del comportamiento de los contratos inteligentes origina en Godoy et. al. \cite{predicate-abstraction-for-smart-contract-validation}.
En esta sección solamente presentamos una compatibilización de los formalismos para facilitar la discusión de las EPAs más adelante.

Dicho eso, llamaremos \textit{configuraciones} a las posibles combinaciones de las variables de estado del contrato y de la blockchain, y notaremos $C$ al conjunto de todas las posibles configuraciones.

\begin{definition}(Formalización de un contrato inteligente)
    \label{definicion-smart-contract}
    Definimos a un contrato inteligente como la tupla $SC = \langle M, F, R, inv, init \rangle$ donde:

    \begin{itemize}
        \item $M = {m_1, \dots m_n}$ es el conjunto finito de métodos externos definidos en la interfaz del contrato
        \item $F$ es un conjunto de funciones indexadas por $M$. \\
              Para cada $m \in M$, $F_m : C \times \mathds{Z} \rightarrow (C \cup \bot)$ es la implementación del método $m$.
        \item $R$ es un conjunto de precondiciones indexado por $M$.\\
              Para cada $m \in M$, $R_m : C \times \mathds{Z} \rightarrow \{$\textbf{true}, \textbf{false}$\}$ indica si el método $m$ está habilitado para la configuración y parámetros indicados
        \item $inv : C \rightarrow \{$\textbf{true}, \textbf{false}$\}$ indica si una configuración cumple el invariante del contrato
        \item $init : C \rightarrow \{$\textbf{true}, \textbf{false}$\}$ indica si una configuración puede ser resultante de ejecutar el constructor del contrato
    \end{itemize}
\end{definition}

Una particularidad presente en los métodos de los contratos inteligentes es que algunas instrucciones como \textcolor{blue}{\texttt{msg.sender}}, \textcolor{blue}{\texttt{msg.value}}, \textcolor{blue}{\texttt{tx.gasprice}} o \textcolor{blue}{\texttt{tx.origin}}
\footnote{\textcolor{blue}{\texttt{msg.value}} indica cuánto Ether está recibiendo el contrato. \textcolor{blue}{\texttt{tx.gasprice}} indica cuánto Ether se le cobra al remitente por cada unidad de gas (cómputo) utilizada y \textcolor{blue}{\texttt{tx.origin}} se refiere a el usuario humano que haya enviado el mensaje que disparó la ejecución actual.} hacen referencia a variables de la blockchain.
Sin embargo podemos modelar estas variables, junto con los parámetros explícitos como input codificable en $\mathds{Z}$ (los números enteros) sin pérdida de generalidad \cite{de-caso-epa}.
La semántica de un contrato la definimos como el siguiente Labeled Transition System:

\begin{definition}\label{definicion-lts}(Semántica de un contrato inteligente)
    Dado $SC = \langle M, F, R, inv, init \rangle$ un contrato inteligente, su semántica está provista por el LTS concreto $L_c = \langle \sigma , S_c, S_{0c}, \Delta _c \rangle$ que satisfaga:
    \begin{itemize}
        \item $S_c = \{conf | conf \in C \land inv(conf) = \textbf{true}\}$
        \item $S_{0c} = \{conf | conf \in S_c \land init(conf) = \textbf{true}\}$
        \item $\sigma = (F \times \mathds{Z}) \cup \tau$ es el conjunto de todos los posibles llamados a funciones, junto con un elemento $\tau$ que representa un cambio en la blockchain que ocurra de manera independiente al contrato
        \item $\Delta _c \subseteq S_c \times \sigma \times S_c$
        \item $\forall s_1,s_2 \in S_c, m \in M, z \in \mathds{Z} . \\ (s_1,(F_m,z),s_2) \in \Delta _c \iff \bigg( R_m(s_1,z) = \textbf{true} \land   F_m(s_1,z) = s_2 \bigg)$ \\
              $(s_1,\tau,s_2) \in \Delta _c \iff$ un cambio independiente en la blockchain puede llevarnos del estado $s_1$ al estado $s_2$
    \end{itemize}
\end{definition}
Notar que para cualquier contrato, el conjunto $S_c$ de estados de su LTS concreto es infinito.
Esto es porque las configuraciones tienen en cuenta variables de la blockchain externas al contrato.
Incluso para contratos donde las configuraciones de las variables internas es finita, siempre habrá infinitas configuraciones de las variables externas.
\\

A la EPA (es decir, el LTS abstracto) la definimos entonces de la siguiente manera:

\begin{definition}\label{definicion-epa}(Enabledness-Preserving-Abstraction) Dado $SC = \langle M, F, R, inv, init \rangle$ un contrato inteligente y $L_c = \langle \sigma , S_c, S_{0c}, \Delta _c \rangle$ su LTS concreto, entonces el LTS asbtracto $L_A = \langle M \cup \hat{\tau} , 2^M, P_0, \Delta _A \rangle$ es una EPA del contrato y $\alpha : S_c \rightarrow 2^M$ es la función de abstracción, donde se cumple que:
    \begin{itemize}
        \item $2^M$ es el conjunto de partes de $M$
        \item $\forall s \in S_c \: . \:
                  \alpha(s) = \{m | m \in M \land \exists z \in \mathds{Z} . R_m(s,z) = \textbf{true}\}$
        \item $P_0 = \{\alpha(s_0) | s_0 \in S_{0c} \}$
        \item $\forall s_1,s_2 \in S_c, m \in M, z \in \mathds{Z} . \\ (s_1,(F_m,z),s_2) \in \Delta _c \Rightarrow (\alpha(s_1),R_m,\alpha(s_2)) \in \Delta _A$ \\
              $(s_1,\tau,s_2) \in \Delta _c \Rightarrow (\alpha(s_1),\hat{\tau},\alpha(s_2)) \in \Delta _A$
    \end{itemize}
\end{definition}

El conjunto de estados de la \textit{EPA} es $2^M$ (el conjunto de partes de las precondiciones).
La \textit{función de abstracción} de los estados $s_c$ del LTS concreto a los estados abstractos es $\alpha (s_c) = $ ``el conjunto de métodos cuyas precondiciones son satisfechas por $s_c$''.
Una transición en la EPA entre los estados $s$ y $s'$ etiquetada con el método $m$ significa que existen algún estado concreto $s_c$ y un valor de entrada $z$ para los que $\alpha(s_c) = s$ y  $F_m(s,z)=s_c'$ con $\alpha(s_c') = s'$.
Es decir que algún llamado de $m$ a un estado que se abstrae a $s$ nos da de resultado otro estado que se abstrae a $s'$.
Una transición de $s$ a $s'$ etiquetada  con $\hat{\tau}$, la versión abstracta de $\tau$, indica que existe un estado concreto $s_c$ con $\alpha(s_c) = s$ y que puede ocurrir $\alpha (\tau (s_c)) = s'$.


\chapter{Construcción de EPAs mediante ejecución simbólica}
No sé que más decir más allá de como implementamos el algoritmo clásico y que la solución implementada pide agregar manualmante las precondiciones en métodos aparte, etc.

\chapter{Análisis}
%A pesar de que Manticore permita el uso de valores simbólicos para el parámetro \texttt{(msg.sender)} las limitaciones de la herramienta exigen que se defina cada account individualmente para que sea considerada en el estado de la blockchain.
Las limitaciones de la herramienta exigen que el parámetro \texttt{(msg.sender)}, a pesar de que pueda ser simbólico, se corresponda con una \textit{account} definida explícitamente por el usuario.
Por esto en la simulación contamos con un número fijo de \textit{accounts}, asignando variables simbólicas únicamente la información asociada a ellas (Dirección, Balance, etc).

\section{Comparación de las abstracciones generadas entre el algoritmo clásico y el alternativo}
Como mencionamos en la sección \ref{sec:subaproximacion}, las abstracciones generadas por el algoritmo alternativo no siempre son Enabledness Preserving Abstractions\footnote{Es decir, en algunos casos no cumplen la definición presentada en la sección \ref{definicion-epa}.} de los contratos analizados, sino que son unas subaproximaciones de estas.
Al mismo tiempo, las EPAs son sobreaproximaciones del comportamiento de los contratos, por lo que el mecanismo resultante del algoritmo alternativo yace en algún lugar intermedio: no es ni sound ni complete.
Por esto, nos interesa explorar en particular cuáles son las diferencias producidas entre el algoritmo alternativo y las generadas por el algoritmo clásico (las EPAs) para evaluar su utilidad.
Para responder estas preguntas analizamos el desempeño de los prototipos implementados usando ambos algoritmos sobre algunos casos de prueba.

\subsection{Caso \texttt{RoomThermostat}}
El primer caso que evaluamos fue el contrato \texttt{RoomThermostat} perteneciente al benchmark ``Microsoft Azure Blockchain Workbench" \cite{azure-benchmark}, cuyo código fuente podemos ver en la figura \ref{fig:rooomthermostat-solidity}.
Es un contrato pequeño, con pocos métodos y un invariante sencillo.
Para poder emplear el algoritmo clásico sobre este contrato, el invariante propuesto fue el siguiente:
\begin{lstlisting}[language=Solidity]
    function invariant(StateType stateNew, address installerNew, address userNew, int targetTemperatureNew, ModeEnum modeNew) public returns(bool){
        bool result = (stateNew == StateType.Created || stateNew == StateType.InUse);
        result = result && (modeNew == ModeEnum.Auto || modeNew == ModeEnum.Cool || modeNew == ModeEnum.Heat || modeNew == ModeEnum.Off);
        if(stateNew == StateType.Created){
            result = (targetTemperatureNew == 70) && (modeNew == ModeEnum.Off);
        }
        return result;
    }
\end{lstlisting}

Luego, evaluamos tanto el algoritmo clásico como el alternativo sobre el contrato.
En ambos casos la EPA generada fue la misma, una máquina de estados sencilla de tan sólo tres estados que podemos ver en la figura \ref{fig:room-thermostat-epa}.

\begin{figure}
    \centering
    \begin{subfigure}{0.75\textwidth}
        \includegraphics[width=\textwidth]{figs/room-thermostate-epa.png}
        \caption{Enabledness Preserving Abstraction del contrato \texttt{RoomThermostat} }
        \label{fig:room-thermostat-epa}
    \end{subfigure}
\end{figure}

\subsection{Caso \texttt{BoundedStack}}
El siguiente caso que evaluamos fue el ejemplo teórico que expusimos en la sección \ref{sec:subaproximacion}, el contrato \texttt{BoundedStack}, con intención de evidenciar las subaproximaiones realizadas por el algoritmo nuevo.
El código del contrato se encuentra en la imagen \ref{code:solidity-bounded-stack}.
Nos interesa corroborar el comportamiento de ambos algoritmos, si se produce alguna subaproximación, y cuáles.

Lo primero que vimos al evaluar la abstracción generada por el algoritmo alternativo sobre el contrato \texttt{BoundedStack} lo podemos ver en la imagen \ref{fig:buggy-bounded-stack-epa}.
En esa abstracción  podemos ver que faltan algunas de las transiciones presentes en la EPA del contrato, pero además podemos observar que las transiciones por \textcolor{orange}{\texttt{pop}} siempre llevan al estado desde el que se ejecutó el método.
Cuando realizamos esta comparación generó bastante confusión, hasta que comprendimos que se debía a un error en el código fuente de \texttt{BoundedStack} que no decrementaba la variable \texttt{size} al ejecutar \textcolor{orange}{\texttt{pop}}.
Luego de corregir el error, la abstracción generada por el prototipo fue la referenciada en la imagen \ref{fig:bounded-stack-bad-epa}.
Es decir, la misma que elaboramos durante el ejemplo teórico del comportamiento del algoritmo alternativo.
Decidimos incluir este pequeña caso de un error en el código fuente en lugar de omitirlo para ejemplificar que, a pesar de que quede evidenciado que el algoritmo nuevo genere subaproximaciones de las EPAs en la práctica, las abstracciones generadas pueden seguir resultando útiles a la hora de atrapar errores.

\begin{figure}[H]
    \centering
    \begin{subfigure}{0.45\textwidth}
        \includegraphics[width=\textwidth]{figs/buggy-bounded-stack-epa.png}
        \caption{Enabledness Preserving Abstraction del contrato \texttt{BoundedStack} con un error que no decrementa el tamaño luego de ejecutar \textcolor{orange}{\texttt{pop}} }
        \label{fig:buggy-bounded-stack-epa}
    \end{subfigure}
\end{figure}

A la hora de emplear el algoritmo clásico sobre el contrato, el invariante propuesto fue el siguiente:
\begin{lstlisting}[language=Solidity]
    function invariant() public returns(bool){
        bool result = (size >= 0) && (size <= maxSize);
        result = result && (internal_arr.length == size);
        return result;
    }
\end{lstlisting}

Sin embargo, el prototipo desarrollado con el algoritmo clásico no pudo analizar este contrato satisfactoriamente, sino que terminó el analisis por time out.
Esto se debe a que el paso de restringir las instancias del contrato a aquellas que satisfagan el invariante resultó demasiado difícil para el motor de ejecucion simbólica de Manticore
Algunos experimentos y análisis intermedios nos hacen creer que se debe a la pobre representación de las variables de tipo \textcolor{cyan}{\texttt{uint256[]}} del mismo.
Esto significó que terminamos el análisis comparando la abstracción generada por el algoritmo alternativo no con la generada por el algoritmo clásico, sino con una EPA producida manualmente.

\section{Comparación en tiempo de ejecución entre el algoritmo clásico y el alternativo}
Otra hipótesis generada durante el desarrollo del algoritmo alternativo fue que el evitar la ejecución de los invariantes mejoraría el tiempo de ejecución.
Esto se debe a que los invariantes, comparadas con las precondiciones de los métodos externos, son propiedades relativamente complejas que además pueden hablar sobre el estado de variables internas de tipos complejos como arrays, mappings, etc.
Para evaluar esta hipótesis estudiamos el tiempo de ejecución empleado por los prototipos que implementan el algoritmo alternativo y el clásico en Manticore sobre algunos de los contratos pertenecientes al benchmark Microsoft Azure Blockchain Workbench \cite{azure-benchmark}.

Ya que este mismo benchmark había sido utilizado por Godoy et al. en 2022 \cite{predicate-abstraction-for-smart-contract-validation}, a la hora de experimentar contamos con EPAs de todos los contratos involucrados, por lo que además pudimos usarlas para corroborar la correctitud de las abstracciones generadas.
Los experimentos fueron realizados ejecutando ambos prototipos sobre cada contrato cinco veces y luego tomando el promedio del tiempo total de ejecución.
Todas las mediciones fueron realizadas en una máquina en una máquina \textcolor{red}{\textbf{DESCRIPCION MAQUINA}}.

En la figura \ref{fig:classic-vs-alternativo} podemos ver los resultados del experimento realizado.
Las mediciones fueron solo realizadas para los contratos \texttt{DefectiveComponentCounter}, \texttt{RoomThermostat}, \texttt{BasicProvenance} y \texttt{SimpleMarketplace} porque como podemos ver el tiempo de ejecución de la implementación del algoritmo clásico nunca fue menor a diez horas.
A pesar de que en todos los contratos analizados podamos ver una diferencia significativa al emplear el algoritmo alternativo, es importante destacar que los tiempos de ejecución de esta técnica aún se mantuvieron muy altos.
La mejora en tiempo parecería ser relativamente constante a lo largo de los cuatro contratos analizados, habiendo tardado el algoritmo alternativo alrededor de diez horas menos que el clásico.
Esta diferencia cercana a constante posiblemente se deba a que todos los contratos eran muy simples (recordemos la baja cantidad de métodos externos en \texttt{SimpleMarketplace} y \texttt{RoomThermostat}), por lo que no se pudo observar la diferencia en comportamiento que habría en contratos más complejos.

A pesar de que la mejora del algoritmo alternativo frente al clásico fuese tan grande, los valores obtenidos con el algoritmo alternativo siguieron sin resultar razonables.
Si observamos los valores obtenidos, el tiempo de cómputo en promedio fue de entre media hasta casi tres horas, lo que no resulta aplicable para un usuario de la herramienta que pretenda generar las abstracciones en tiempo real.
Por este motivo decidimos realizar los experimentos que mostraremos a continuación, en los que buscamos conseguir mejoras en el tiempo de ejecución o claridad en la razón de que tarde tanto.

\begin{figure}
    \centering
    \begin{subfigure}{0.65\textwidth}
        \includegraphics[width=\textwidth]{figs/classic_vs_alternativo.png}
        \caption{Tiempo de ejecución (en promedio) de la generación de abstracciones del prototipo que implementa el algoritmo alternativo y el algoritmo clásico}
        \label{fig:classic-vs-alternativo}
    \end{subfigure}
\end{figure}

\section{Evaluación más detallada del tiempo de ejecución del algoritmo alternativo}
Para mejorar el tiempo de ejecución del prototipo buscamos identificar en qué partes del análisis era que se consumía la mayor parte del tiempo empleado.
Lo primero que hicimos fue identificar ``etapas'' en el algoritmo alternativo que pudimos diferenciar para medir el tiempo empleado en cada una de ellas.
Por otro lado analizamos el código fuente de Manticore y diferenciamos en distintos niveles de abstracción las funcionalidades que estábamos usando.

Separamos la ejecución del prototipo en tres etapas:
\begin{enumerate}
    \item El desplegado simbólico del contrato
    \item La ejecución simbólica de los métodos
    \item La solución de queries de satisfacibilidad sobre las transiciones en la EPA
\end{enumerate}
Donde notoriamente el algoritmo alternativo comienza en la etapa 1., y luego alterna entre la etapa 3. y la 2. hasta finalizar.

A continuación indicamos cuál fue la división en niveles de abstracción que hicimos de las funcionalidades de Manticore.
En particular, por como es la arquitectura de multithreading y \textbf{\texttt{workers}} de Manticore, solo pudimos hacer esta distinción en los métodos relacionados a la resolución de valores simbólicos.
La manera en la que se realizaba la ejecución simbólica y la generación de las path conditions era demasiado dependiente de objetos creados dinámicamente como para poder recibir el mismo tipo de análisis.
\begin{enumerate}
    \item \textbf{Nivel Externo}\footnote{En el momento de realizar las mediciones, los valores para este nivel se registraron bajo el nombre de  ``Nivel 3''} : utilizado para medir el tiempo que consumían las funciones de la API de manticore que nuestro prototipo llamaba directamente.
          La única función cuyo tiempo medimos en este nivel fue \texttt{generateTestcases}.
    \item \textbf{Nivel \texttt{state}} : Utilizado en las funciones del módulo \textbf{\texttt{state}} que identificamos que eran llamadas.
          Los métodos registrados bajo este nivel fueron \texttt{can\_be\_true} y \texttt{solve\_\allowbreak one\_\allowbreak n\_\allowbreak batched}.
    \item \textbf{Nivel SMT solver} : Utilizado en alrededor de cada llamado directo al SMT solver en los métodos del módulo \textbf{\texttt{smtlib}}.
          Los métodos registrados bajo este nivel fueron \texttt{\_is\_sat}, \texttt{\_get\_value} y \texttt{\_\_get\_value\_all}.
\end{enumerate}
Cada uno de los métodos fue modificado para medir el tiempo desde el principio de la zona de interés hasta el fin de la misma.
Por ejemplo, en la figura \ref{code:getvalueall-modification} podemos ver los cambios introducidos al método \texttt{\_\_get\_value\_all} para registrar el tiempo.

\begin{lstlisting}[language=Python,
    label={code:getvalueall-modification},
    caption=\text{Método \texttt{\_\_get\_value\_all} del modulo \textbf{\texttt{smtlib}} modificado para registrar el tiempo empleado por la llamada al SMT solver. El resaltado indica las líneas agregadas para registrar el tiempo.},
    captionpos=b]
def __getvalue_all(self, expressions_str: List[str], is_bv: List[bool]) -> Dict[str, int]:
    (*@| \hl{start = time.time()} |@*)
    all_expressions_str = " ".join(expressions_str)
    self._smtlib.send(f"(get-value ({all_expressions_str}))")
    ret_solver: Optional[str] = self._smtlib.recv()
    (*@| \hl{print}|@*)(f"(level _getvalue_all_z3_call) took {time.time()- start} seconds")
    assert ret_solver is not None
    return_values = re.findall(RE_GET_EXPR_VALUE_ALL, ret_solver)
    
    return {value[0]: _convert(value[1]) for value in return_values}
\end{lstlisting}

De  entre los métodos a los que les registramos el tiempo de ejecución, sabíamos que los métodos en \textbf{Nivel Externo} solo eran ejecutados cuando el prototipo implementado los llamaba explícitamente.
Por otro lado, los métodos en \textbf{Nivel \texttt{state}} eran métodos que a veces eran llamados dsde la implementación directamente, pero además eran ejecutados como métodos internos de otros procesos durante el análisis.
Por último, los métodos del \textbf{Nivel SMT solver} eran ejecutados en diversos momentos del análisis.

Una vez hecha esta separación en etapas del algoritmo y niveles de abstracción que nos resultaba de interés investigar, medimos el tiempo total en promedio empleado por cada etapa y cada nivel de abstracción en una ejecución del prototipo.
Para esto, al igual que antes, analizamos el rendimiento en los contratos \texttt{DefectiveComponentCounter}, \texttt{RoomThermostat}, \texttt{BasicProvenance} y \texttt{SimpleMarketplace} del benchmark Microsoft Azure Blockchain Workbench \cite{azure-benchmark}.
Por otro lado, además, calculamos para cada uno de los niveles de abstracción diferenciados la cantidad de veces que este mismo fue registrado y el tiempo total empleado.
Esto fue para intentar percibir discrepancias entre los niveles que espérabamos que consumieran la mayoría del tiempo y lo que pudiéramos medir.

En la figura \textcolor{red}{\textbf{dani hace la figura escribila}} podemos ver los resultados obtenidos de estas mediciones.


Al evaluar el funcionamiento del prototipo buscamos responder las siguientes preguntas:
\begin{enumerate}
    \item ¿Son correctas las EPAs que genera el prototipo?
    \item ¿Cuál es su performance en contratos inteligentes reales?
\end{enumerate}
Para responder estas preguntas, pusimos el prototipo a prueba contra algunos contratos provenientes de Microsoft Azure Blockchain Workbench \cite{azure-benchmark}.
Este conjunto de contratos había sido utilizado anteriormente por Godoy et al. \cite{predicate-abstraction-for-smart-contract-validation}, por lo que convenientemente ya contamos con las EPAs correspondientes.
Ejecutamos el prototipo cinco veces, obteniendo el promedio de su tiempo de ejecución sobre los contratos seleccionados y luego corroboramos que la EPA generada fuera isomórfica a la obtenida en los estudios anteriores.
%Además, pusimos a prueba la cantidad de accounts necesarias para el funcionamiento del prototipo.
Los resultados intermedios indicaron que para los contratos propuestos era suficiente realizar la simulación con dos \textit{accounts}, por lo que utilizamos esa cantidad para los experimentos.
Los resultados de esta experimentación se ven resumidos en la tabla \ref{tab:resultados}.
El prototipo generó EPAs correctas en todos los casos.
Sin embargo, el tiempo de cálculo es de entre media y casi tres horas, considerando incluso que los ejemplos utilizados son relativmente pequeños.
Algunos resultados intermedios indican que este tiempo es consumido principalmente por Manticore para la generación de \textit{path conditions}.
Debido a la rigurosidad con la que emula Manticore el comportamiento de la blockchain, la herramienta demora incluso para la ejecución simbólica de transiciones sencillas.

\vspace{-2.2em}

\begin{table}
    %\begin{adjustwidth}{-0.5in}{-0.5in}% adjust the L and R margins by 1 inch
    \caption{Resumen de la experimentación. \textbf{LOC} es cantidad de lineas de código, \textbf{Tiempo de ejecución} es el promedio del tiempo de ejecución medido en minutos, $\boldsymbol{\sigma}$ es el desvío estandard medido en segundos y \textbf{¿Es correcto?} indica si la EPA generada es isomorfa con la provista anteriormente.}\label{tab:resultados}
    \begin{tabular}{|l|l|l|l|l|}
        \hline
        Contrato                  & LOC & Tiempo de ejecución (min) & $\sigma$ (s) & ¿Es correcto? \\
        \hline
        DefectiveComponentCounter & 33  & $29$                      & $7$          & Sí            \\
        SimpleMarketplace         & 66  & $186$                     & $800$        & Sí            \\
        BasicProvenance           & 48  & $40$                      & $6$          & Sí            \\
        RoomThermostat            & 48  & $138$                     & $600$        & Sí            \\
        \hline
    \end{tabular}
    %\end{adjustwidth}
\end{table}


\chapter{Conclusiones}

%%%% BIBLIOGRAFIA
\backmatter
\printbibliography


\end{document}
